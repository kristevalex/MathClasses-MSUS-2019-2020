%\author{Олег Смирнов}

\documentclass[12pt, a4paper]{article}
%\usepackage[9pt]{extsizes}

\usepackage[top=30pt, left=35pt, right=35pt, bottom=10pt]{geometry}
\geometry{a4paper,portrait}

\usepackage{myincludes}
\usepackage{mypack}

\begin{document}

\setcounter{iii}{0}

\begin{center}
	\textbf{Вступительный тест}
\end{center}

\ii Что больше: $343^{33}$ или $49^{50}$?

\ii Сколько остатков дают точные квадраты при делении на 60?
%12

\ii А какие остатки дают точные 6-е степени при делении на 13?
%0, 1, 12

\ii Найдите количество слов \textit{(словом считается любая последовательность букв)}, которые можно составить из букв В, О, Д, О, Р, О, Д. 
%60

\ii Найдите количество собрать из стандартной колоды карт (4 масти по 13 номиналов) червовый флеш.
%1287

\ii Дан граф на 4 вершинах и 5 ребрах. Найдите количество его остовных деревьев.
%8

\ii В одной столичной школе дети интересуются математикой, балетом и киберспортом. Директор этой школы заметил, что среди тех, кто любит математику, $1/30$ нравится балет и $5/6$ неравнодушны к компьютерным играм. Те, кто занимается балетом, в 40% случаев любят математику и на 25% геймеры. Какая наибольшая доля киберспортсменов может увлекаться балетом?
%$\frac{1}{40}$

\ii Дана куча из 2019 камней. Алиса и Боб (начинает Алиса) играют в игру: за ход разрешается взять из кучи любое количество камней, являющееся степенью двойки (то есть $1, 2, 4, \dots$). Выигрывает тот, кто возьмёт последний камень. Кто из игроков может выиграть, независимо от действий соперника?
%Боб

\ii Пусть $x$, $y$ и $z$ - различные корни уравнения $x^3+8=5x^2$. Найдите $\frac{1}{x}+\frac{1}{y}+\frac{1}{z}$.
%0

\ii Найдите $\frac{1}{ \{ \frac{1}{ 3\sqrt{2} - 4 } \} }$. \textit{( $\{ x \}$ - дробная часть числа $x$.)}
%$4+3\sqrt{2}$

\ii Что больше: $e^{\frac{1}{e}}$ или $\pi^{\frac{1}{\pi}}$?
%$e^{\frac{1}{e}}$

\ii На сторонах $CA$ и $CB$ треугольника $ABC$ отмечены точки $P$ и $Q$ соответственно так, что $CP:PA = 1:2$ и $CQ:QB=2:1$. Отрезки $AQ$ и $BP$ пересекаются в точке $O$. Найдите отношение площадей треугольников $APO$ и $BQO$.
%$1:8$

\ii Дан тетраэдр $ABCD$. Пусть $\Gamma$ - его описанная сфера. Обозначим за $A_1$ точку на $\Gamma$, диаметрально противоположную $A$, за $M_A$ - точку пересечения медиан треугольника $BCD$, и за $l_A$ - прямую $A_1 M_A$. Прямые $l_B$, $l_C$ и $l_D$ определяются аналогично.
\Pu Докажите, что $l_A$, $l_B$, $l_C$, $l_D$ пересекаются в одной точке.
\Pu Пусть $O$ - центр $\Gamma$, $M$ - центроид тетраэдра (точка пересечения $AM_A$, $BM_B$, $CM_C$ и $DM_D$). Докажите, что точка из предыдущего пункта лежит на $OM$.
 
\vfill
	
\end{document}
