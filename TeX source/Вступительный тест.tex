









































%\author{Олег Смирнов}

\documentclass[12pt, a4paper]{article}
%\usepackage[9pt]{extsizes}

\usepackage[top=30pt, left=35pt, right=35pt, bottom=10pt]{geometry}
\geometry{a4paper,portrait}

\usepackage{myincludes}
\usepackage{mypack}

\begin{document}

\setcounter{iii}{0}

\begin{center}
	\textbf{Блок <<Теория чисел>>.}
\end{center}

\ii Что больше: $343^{33}$ или $49^{50}$?

\ii Какие остатки дают точные квадраты при делении на 13?
%0, 1, 3, 4, 9, 10, 12

\ii А какие остатки дают точные 6-е степени при делении на 13?
%0, 1, 12

\ii Найдите количество слов \textit{(словом считается любая последовательность букв)}, которые можно составить из букв В, О, Д, О, Р, О, Д. 
%60

\ii Дан граф на 4 вершинах и 5 ребрах. Найдите количество его остовных деревьев.
%8

\ii В одной столичной школе дети интересуются математикой, балетом и киберспортом. Директор этой школы заметил, что среди тех, кто любит математику, $1/30$ нравится балет и $5/6$ неравнодушны к компьютерным играм. Те, кто занимается балетом, в 40% случаев любят математику и на 25% геймеры. Какая наибольшая доля киберспортсменов может увлекаться балетом?
%$\frac{1}{40}$

\ii Пусть $x$, $y$ и $z$ - различные корни уравнения $x^3+8=5x^2$. Найдите $\frac{1}{x}+\frac{1}{y}+\frac{1}{z}$.
%0

\ii Найдите $\frac{1}{ \{ \frac{1}{ 3\sqrt{2} - 4 } \} }$. \textit{( $\{ x \}$ - дробная часть числа $x$.)}
%$4+3\sqrt{2}$

\ii Что больше: $e^{\frac{1}{e}}$ или $\pi^{\frac{1}{\pi}}$?
%$e^{\frac{1}{e}}$

\ii На сторонах $AB$ и $AC$ треугольника $ABC$ отмечены точки $Z$ и $Y$ соответственно так, что $AZ:ZB = 1:2$ и $AY:YC=2:1$. Отрезки $BY$ и $CZ$ пересекаются в точке $O$. Прямая $AO$ пересекает $BC$ в точке $X$. Найдите отношение $XO:OA$.
%$2:5$

\ii Даны окружности $\omega_1$ и $\omega_2$, которые пересекают третью окружность $\Gamma$ по точкам $A$, $B$, $C$ и $D$. $KM$ и $LN$ - общие внешние касательные к $\omega_1$ и $\omega_2$. ($A,B,K,L \in \omega_1$, $C,D,M,N \in \omega_2$.) Докажите, что середина $KM$, середина $LN$ и точка пересечения $AB$ и $CD$ лежат на одной прямой.

\vfill
	
\end{document}

