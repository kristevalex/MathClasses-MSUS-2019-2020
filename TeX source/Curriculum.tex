%\author{Олег Смирнов}

\documentclass[12pt, a4paper]{article}
%\usepackage[9pt]{extsizes}

\usepackage[top=30pt, left=35pt, right=35pt, bottom=10pt]{geometry}
\geometry{a4paper,portrait}

\usepackage{myincludes}
\usepackage{mypack}

\begin{document}

\begin{center}
\textbf{
	Учебный план программы "Олимпиадная Математика"
	}
,\parсегмента курсов "Вершина Экономики"\ при МГУУ Правительства Москвы.
\end{center}

\setcounter{iii}{0}

\ii Вступительный тест. \textit{Большое количество заданий, проверяющие владение конкретной техникой или знание конкретного факта.}

\ii Алгебра. Системы уравнений. Симметризация. Advanced: Графическая интерпретация уравнения $f(f(f(x)))=x$.

\ii Теория чисел. Сравнения по модулю, возведение остатков в степень. Advanced: МТФ \textit{(несколько доказательств и задачи на неё)}.

\ii Комбинаторика. Графы. Чётность. Остовные деревья. Базовое представление о двудольных графах. Advanced: Лемма Холла и её применения.

\ii Теория чисел (Продолжение). Операции с модулями \textit{(практика решения ТЧ-шных задач)}. Advanced: Дроби как остатки по модулю. Теорема Вильсона.

\ii Алгебра. Теорема Виета. Advanced: Интерполяция по Лагранжу ($P(i)=2^i$ при $i=0,\dots,k$, найдите $P(k+1)$).

\ii Комбинаторика. Теория вероятности. Биномиальные коэффициенты. \textit{(Количество способов собрать покерные комбинации.)} Круги Эйлера, формула включений и исключений. Advanced: Задача об ожерельях.

\ii Геометрия. Аффинная геометрия. Теоремы Чевы и Менелая. \textit{(В т.ч. и задачи высшего уровня сложности.)}

\ii Алгебра + комбинаторика. Бином Ньютона, альтернативная трактовка биномиальных коэффициентов. Комбинаторные доказательства алгебраических фактов. Advanced: Числа Каталана.

\ii Комбинаторика на плоскости. Куча задач прикольных. Advanced: Планарные графы. Формула Эйлера.

\ii Теория чисел. Простые числа, степени вхождения. НОДы и НОКи. Часто встречаемые сюжеты с разложением на множители и решением через простые числа.

\ii Экстремальные текстовые задачи. Вершинка параболы, экономические задачи. Advanced: Неравенства о средних.

\ii Математический анализ. Ликбез по производным. Олимпиадные задачи с элементами функционального анализа.

\ii Многочлены над $\mathbb{Z}$. Задачи, использующие утверждение $(x-y) | P(x)-P(y)$. Advanced: Критерий Эйзенштейна.
$\mathbb{Z}$. Задачи, использующие утверждение $(x-y) | P(x)-P(y)$. Advanced: Критерий Эйзенштейна.

\ii Основы рекуррентных соотношений. Идея, которая появляется при решении рекурренты $\{ a_{i+1}=2a_i-1 \}$: линейный сдвиг последовательности может упростить рекуррентное соотношение.

\ii Иррациональные числа. Домножение на сопряженное. $\{ x \} = x - [x]$.

\ii Нетривиальные задачи на движение. Минимизация суммы расстояний до точек на прямой. Advanced: Задачи на линейное движение по плоскости.

\ii Тригонометрия. Системы уравнений, в которых надо ввести тригонометрические функции.

\ii Телескопические суммы. Задачи на суммирование прогрессии и часто используемые трюки.

\ii Клетчатые задачи. Раскраски \textit{(В т.ч. полный перечень типовых раскрасок)}, нетривиальные рассуждения с их использованием.

\ii Стереометрия. Нахождение кратчайшего пути \textit{(на поверхностях многогранников и конусов)} путём "распрямления" поверхности.

\newpage

\begin{center}
\textbf{
	Домашние задания, цель которых - устранить пробелы и помочь "набить руку"\ на базовых вещах.
}
\end{center}

\setcounter{iii}{0}

\ii Доказательство формул методом математической индукции.

\ii Построение графиков функций. Функции с модулем. Графическая интерпретация в задачах с параметром.

\ii Упражнения, помогающие освоиться в мире сравнений по модулю.

\ii Аффинная геометрия, выражение одних отношений отрезков на картинке через другие.

\ii Тренировка использования теоремы Виета. Умение выражать всё через симметрические многочлены.

\ii Базовые задачи на круги Эйлера.

\ii Базовые задачи на степени вхождения простых.

\ii Тождественные преобразования, тренировка аккуратности (как не терять и не находить лишние корни).

\ii Упражнения на операции с иррациональными числами.

\ii Дифференцирование функций. Нахождение максимума и минимума функции.

\ii Тригонометрические формулы, как запомнить их без "зубрёжки".

\ii Преобразования со степенями и логарифмами. Как не запутаться в том, когда складывать, а когда перемножать.

\ii Применение формулы суммы геометрической прогрессии на практике.

\vfill

\begin{flushright}
    Документ подготовлен Кристевым Алексеем и Смирновым Олегом.
\end{flushright}
	
\vspace{15pt}

\end{document}
